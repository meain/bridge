\documentclass{article}

\usepackage{titlesec}
\usepackage[margin=1in]{geometry}

\titleformat{\section}
{\huge}
{}
{0em}
{\bfseries}[\titlerule]

\titleformat{\subsection}
{\LARGE}
{}
{0em}
{\bfseries}

\titleformat{\subsubsection}
{\Large}
{}
{0em}
{\bfseries}

\renewcommand{\maketitle}{
\begin{center}
\LARGE
Mini Project Report

\hspace{1em}

on

\hspace{1em}

\huge
BRIDGE

\hspace{1em}
\Large
Student Self Management System

\hspace{1em}

submitted by

\hspace{1em}

\Large
Abin Simon ( 12150005 )

\Large
Aayisha A A ( 12150076 )

\Large
Abhai Kollara ( 12150002 )
\end{center}
}

\begin{document}

\title{Bridge}

\topskip0pt
\vspace*{\fill}

\maketitle

\begin{center}
\large
In partial fulfilment of the requirements for the award of degree of Bachelor of Technology in Computer Science and Engineering.

\hspace{1em}

\LARGE
DIVISION OF COMPUTER ENGINEERING 

SCHOOL OF ENGINEERING

COCHIN UNIVERSITY OF SCIENCE AND TECHNOLOGY

\hspace{1em}

\large
MARCH 2017
\end{center}

\vspace*{\fill}

\newpage

\topskip0pt
\vspace*{\fill}

\begin{center}

\LARGE
DIVISION OF COMPUTER ENGINEERING\\
SCHOOL OF ENGINEERING\\
COCHIN UNIVERSITY OF SCIENCE AND TECHNOLOGY\\

\hspace{1em}

\huge
\emph{CERTIFICATE}

\hspace{1em}

\large
Certified that this is a bonafide record of the Minor Project titled

\hspace{1em}

\LARGE
BRIDGE

\large
Student Self Management System

\hspace{1em}

done by

\hspace{1em}

\Large
Abin Simon ( 12150005 )

\Large
Aayisha A A ( 12150076 )

\Large
Abhai Kollara ( 12150002 )

\hspace{1em}

\large
of VI Semester, Computer Science and Engineering in the year 2017 in partial fulfillment requirements for the award of degree of Bachelor of Technology in Computer Science and Engineering of Cochin University of Science and Technology.

\hspace{1em}
\vspace{5em}

\begin{minipage}[b]{0.33333\textwidth}
\raggedright
Ancy Zachariah

Head of Division\\
\end{minipage}%
\begin{minipage}[b]{0.33333\textwidth}
\centering
Pramod Pavithran / Damodaran.V

Project Coordinator\\
\end{minipage}%
\begin{minipage}[b]{0.33333\textwidth}
\raggedleft
Anu Ajith

Project Guide\\
\end{minipage}



\end{center}

\vspace*{\fill}

\newpage

% \vspace*{\fill}

\section{Acknowledgement}
\vspace{1em}

\Large
We take this opportunity to express our profound gratitude and deep regards to our guide \textbf{Mrs Anu M.}
,for providing us with the right guidance and advice at the crucial junctures and for her constant encouragement throughout the course of this project. We are highly indebted to \textbf{Asst. Prof. Pramod Pavithran}
 and \textbf{Mr Damodaran.V}
 , Division of Computer Science our batch coordinator for her constant supervision and support for completing the project. We extend our sincere thanks to our respected Head of the department, \textbf{Ancy Zachariah}
 ,Head of Division,Division of Computer Science and all other faculty members of for sharing their valuable time and knowledge with us. We thank God, the almighty for blessing us with his grace and taking our endeavour to a successful culmination. Lastly we would like to thank my friends and family for their constant encouragement without which this project would not have been possible.
 
 \begin{flushright}
 
  \vspace{2em}
  
 \Large
 Thanking you,\\
 \vspace{1em}
 
 \Large
 Abin Simon\\
 Aayisha A A\\
 Abhai Kollara\\
 
 \end{flushright}

% \vspace*{\fill}

\newpage

\section{Decleration}
\vspace{1em}
\Large
We, Miss Aayisha A A,Mr Abhai Kollara Dilip,Mr Abin Simon hereby declare that this project is the record of authentic work carried out by us during the academic year 2016 -2017 and has not been submitted to any other University or Institute towards the award of any degree.

\vspace{5em}

\begin{minipage}[b]{0.33333\textwidth}
\raggedright
Abin Simon
\end{minipage}%
\begin{minipage}[b]{0.33333\textwidth}
\centering
Aayisha A A
\end{minipage}%
\begin{minipage}[b]{0.33333\textwidth}
\raggedleft
Abhai Kollara
\end{minipage}

\newpage

\section{Abstract}
\vspace{1em}

\newpage

\section{Introduction}
\hspace{1em}

\Large
The most important aspect of a academic studies for a student is the involvement of the teachers in the matters of the students. It is also very vital for the student to stay updated about what is happening in the college. With this project we aim to do just that. We are creating a platform that can help the students to easily stay up to date on what is happening in their academic matters. They have an easy and intuitive way to see what is in their calendar.\\

\vspace{1em}
\Large
Our platform also provides an effective way for the student to track his daily activities at the academic institute by giving then a simple and efficient platform to take notes and track their attendance. This, we think is a great tool that will help the student of the academic institution to be able to track his time at the institution.\\

\vspace{1em}
\Large
The platform with its ability to take down notes for the classes they are attending in the university, in a platform that has their timetable and other data integrated along with it is a huge bonus for the student as it gives them a seamless way to take down notes for a specific period and let them filter it and get all the dada beautifully presented to them which is a huge bonus for a student.

\newpage


\newpage

\section{System analysis}
\hspace{1em}

\subsection{Existing system}

\Large
The present system is ineffective in maintaining a student centric model. It mainly consisted of web applications and very few mobile applications. The traditional way of maintaining record is time consuming, not easily accessible ,requires a computer, less user friendly, and laborious.\\

\Large
Apart from these all the system we have currently are mostly teacher centric which leads to a hard time for the students to navigate and find what they need.\\

\Large
Here are some disadvantages of existing system:
\begin{itemize}
\item Time consuming
\item Not easily accessible
\item Teacher centric
\item Less student friendly
\item Laborious
\end{itemize}

\vspace{1em}

\subsection{Proposed system}

\Large
With our solution we are aiming to provide a simple and intuitive interface for the student. We are also looking forward to provide a student centric model in which the student will able to track his classes at the university.\\

\Large
Once the user get registered through their google sign in (which makes the signin and verification process efficient and easy ). The user will be provided with a popup menu in which they could select their respective division head and to which course they are enrolled to.after the completion of sign in process, we will have a wide range of options which include attendance tracking system, notes adding, which could include lecture videos, images and much more feature which make learning easier, it tend to have an added feature of upcoming events which help to clear the backlogs and make the work as timid and as clean as possible\\

\begin{itemize}
\item This system is developed in such a way that even a naive user can also operate the system easily.
\item This system is also secure as the database is managed only by
the administrator of the system.
\item It has an error free verification mechanism
\item Easy way of accessing records and tracking attendance
\end{itemize}


\newpage

\section{System study}
\vspace{1em}

\subsection{Software Requirements Specification}
\vspace{1em}
%\subsection{System Objectives}
%\vspace{1em}
\subsection{Hardware and Software requirements}
\vspace{1em}
\subsubsection{Hardware specification}
\vspace{1em}
\begin{itemize}
\item RAM: Recommended 512MB or above
\item Storage: 10 GB or above
\item Connectivity: Low latency internet with high bandwidth
\end{itemize}
\subsubsection{Software specification}
\begin{itemize}
\item OS: Windows, OSX, or Linux ( or any unix system )
\item Env: Python 2.x
\item Database: MySQL
\item Web browser ( preferable Chrome or Firefox )
\end{itemize}
\vspace{1em}

\vspace{1em}

\newpage

\section{System Design}
\vspace{1em}

\subsection{Data Flow Diagrams}
\vspace{1em}
\subsection{Database Design}
\vspace{1em}
\subsection{Modular Design}

\subsubsection{Login}
This module handles the user authentication and login.
The component is dependent on Google login for its authentication, as we leverage Google signin module to make the tradition of the user into our app much more easier and faster.

Initially when the user logs into the application is when we create an account for them and register them. When the user logs into the application, we get the user to login to their Google account and we can use the Google API top get the user details like user-id, name, email, profile-image etc. This makes it easer for the user as do not have to manually add in their name or profile image. Only thing they will have to manually set is which class they are studying in.

Now from the next time the user logs in, we will directly log them into their account ready to go.

\subsubsection{Landing Page}
The landing page is the most important and most viewed part of the whole application. It encompasses details a user would probably need at that time like upcoming events, which data about the class that is currently going on. It also lets them take down notes which they can later view using the notes module.

For each subject of the day we provide the name of the teacher, time of the class, etc which helps the student to plan for their class.
The place provided to jot down notes for the subject also helps them to take down notes during the classes which will be really useful for them for later reference.

\subsubsection{Notes}
Notes module is aimed at providing an interface for the student( user ) to view all the notes they have taken in the class and go through them.
It is a very useful and powerful utility at the end of a semester as it will let them view all the notes they have in one place and go through them quickly and efficiently.

You can also use the notes view to filter your notes based on the date or the subject which is a really simple but powerful way to summarize a whole semester.

\subsubsection{Attendance}
One another very important module is the Attendance module which lets the student track their attendance for each subject they have. The student on attending or net attending a specific subject class can update their attendance using the attendance module. All the user data is saved in real-time with the backed and they will be able to check their any time.

The per subject nature of the attendance module also helps them to know which subjects they have missed the most and concentrate on them individually.

\subsubsection{Calendar}
The Calendar module is used to view the overview of all the events. It shows you all the events and submissions you have for the future. While the upcoming events module only shows you the close and upcoming events, in the calendar module we can see all the upcoming events in the future.

It has a normal calender like intercase with events listed under the specific dates which makes it much more intuitive and easy.

\vspace{1em}
\subsection{Input/Output Design}
\vspace{1em}


\newpage

\section{System Implementation}
\vspace{1em}

\subsection{Sample code}
\vspace{1em}
\subsection{Screenshots}
\vspace{1em}


\newpage


\section{System Testing}
\vspace{1em}


\newpage

\section{Future Scope}
\vspace{1em}

\newpage

\section{Conclusion}
\vspace{1em}

\newpage


\section{Reference}
\vspace{1em}


\newpage


\end{document}
